% Options for packages loaded elsewhere
\PassOptionsToPackage{unicode}{hyperref}
\PassOptionsToPackage{hyphens}{url}
%
\documentclass[
]{book}
\usepackage{lmodern}
\usepackage{amsmath}
\usepackage{ifxetex,ifluatex}
\ifnum 0\ifxetex 1\fi\ifluatex 1\fi=0 % if pdftex
  \usepackage[T1]{fontenc}
  \usepackage[utf8]{inputenc}
  \usepackage{textcomp} % provide euro and other symbols
  \usepackage{amssymb}
\else % if luatex or xetex
  \usepackage{unicode-math}
  \defaultfontfeatures{Scale=MatchLowercase}
  \defaultfontfeatures[\rmfamily]{Ligatures=TeX,Scale=1}
\fi
% Use upquote if available, for straight quotes in verbatim environments
\IfFileExists{upquote.sty}{\usepackage{upquote}}{}
\IfFileExists{microtype.sty}{% use microtype if available
  \usepackage[]{microtype}
  \UseMicrotypeSet[protrusion]{basicmath} % disable protrusion for tt fonts
}{}
\makeatletter
\@ifundefined{KOMAClassName}{% if non-KOMA class
  \IfFileExists{parskip.sty}{%
    \usepackage{parskip}
  }{% else
    \setlength{\parindent}{0pt}
    \setlength{\parskip}{6pt plus 2pt minus 1pt}}
}{% if KOMA class
  \KOMAoptions{parskip=half}}
\makeatother
\usepackage{xcolor}
\IfFileExists{xurl.sty}{\usepackage{xurl}}{} % add URL line breaks if available
\IfFileExists{bookmark.sty}{\usepackage{bookmark}}{\usepackage{hyperref}}
\hypersetup{
  pdftitle={TCR-PTF},
  pdfauthor={PTF},
  hidelinks,
  pdfcreator={LaTeX via pandoc}}
\urlstyle{same} % disable monospaced font for URLs
\usepackage{longtable,booktabs}
\usepackage{calc} % for calculating minipage widths
% Correct order of tables after \paragraph or \subparagraph
\usepackage{etoolbox}
\makeatletter
\patchcmd\longtable{\par}{\if@noskipsec\mbox{}\fi\par}{}{}
\makeatother
% Allow footnotes in longtable head/foot
\IfFileExists{footnotehyper.sty}{\usepackage{footnotehyper}}{\usepackage{footnote}}
\makesavenoteenv{longtable}
\usepackage{graphicx}
\makeatletter
\def\maxwidth{\ifdim\Gin@nat@width>\linewidth\linewidth\else\Gin@nat@width\fi}
\def\maxheight{\ifdim\Gin@nat@height>\textheight\textheight\else\Gin@nat@height\fi}
\makeatother
% Scale images if necessary, so that they will not overflow the page
% margins by default, and it is still possible to overwrite the defaults
% using explicit options in \includegraphics[width, height, ...]{}
\setkeys{Gin}{width=\maxwidth,height=\maxheight,keepaspectratio}
% Set default figure placement to htbp
\makeatletter
\def\fps@figure{htbp}
\makeatother
\setlength{\emergencystretch}{3em} % prevent overfull lines
\providecommand{\tightlist}{%
  \setlength{\itemsep}{0pt}\setlength{\parskip}{0pt}}
\setcounter{secnumdepth}{5}
\usepackage{booktabs}
\usepackage{amsthm}

\makeatletter
\def\thm@space@setup{%
  \thm@preskip=8pt plus 2pt minus 4pt
  \thm@postskip=\thm@preskip
}
\makeatother

\linespread{1.4}
\XeTeXlinebreakskip = 0pt plus 1pt  
\setmainfont{TH Sarabun New:script=thai}[Scale=1.4]
\ifluatex
  \usepackage{selnolig}  % disable illegal ligatures
\fi
\usepackage[]{natbib}
\bibliographystyle{plainnat}

\title{TCR-PTF}
\author{PTF}
\date{2020-11-15}

\begin{document}
\maketitle

{
\setcounter{tocdepth}{5}
\tableofcontents
}
\hypertarget{uxe11uxe35uxe06uxe19uxe34uxe01uxe32uxe22}{%
\chapter*{ฑีฆนิกาย}\label{uxe11uxe35uxe06uxe19uxe34uxe01uxe32uxe22}}
\addcontentsline{toc}{chapter}{ฑีฆนิกาย}

\hypertarget{ds}{%
\chapter{สีลกฺขนฺธวคฺคปาฬิ}\label{ds}}

\hypertarget{uxe1euxe3auxe23uxe2buxe3auxe21uxe0auxe32uxe25uxe2auxe38uxe15uxe3auxe15uxe4d}{%
\section{พฺรหฺมชาลสุตฺตํ}\label{uxe1euxe3auxe23uxe2buxe3auxe21uxe0auxe32uxe25uxe2auxe38uxe15uxe3auxe15uxe4d}}

\hypertarget{uxe1buxe23uxe34uxe1euxe3auxe1euxe32uxe0auxe01uxe01uxe16uxe32}{%
\subsection{ปริพฺพาชกกถา}\label{uxe1buxe23uxe34uxe1euxe3auxe1euxe32uxe0auxe01uxe01uxe16uxe32}}

เอวมฺเม สุตํ ฯ เอกํ สมยํ ภควา อนฺตรา จ ราชคหํ อนฺตรา จ นาฬนฺทํ อทฺธานมคฺคปฏิปนฺโน โหติ มหตา ภิกฺขุสํเฆน สทฺธึ ปญฺจมตฺเตหิ ภิกฺขุสเตหิ ฯ สุปฺปิโยปิ โข ปริพฺพาชโก อนฺตรา จ ราชคหํ อนฺตรา จ นาฬนฺทํ อทฺธานมคฺคปฏิปนฺโน โหติ สทฺธึ อนฺเตวาสินา พฺรหฺมทตฺเตน มาณเวน ฯ ตตฺร สุทํ สุปฺปิโย ปริพฺพาชโก อเนกปริยาเยน พุทฺธสฺส อวณฺณํ ภาสติ ธมฺมสฺส อวณฺณํ ภาสติ สํฆสฺส อวณฺณํ ภาสติ ฯ สุปฺปิยสฺส ปน ปริพฺพาชกสฺส อนฺเตวาสี พฺรหฺมทตฺโต มาณโว อเนกปริยาเยน พุทฺธสฺส วณฺณํ ภาสติ ธมฺมสฺส วณฺณํ ภาสติ สํฆสฺส วณฺณํ ภาสติ ฯ อิติห เต อุโภ อาจริยนฺเตวาสี อญฺญมญฺญสฺส อุชุวิปจฺจนิกวาทา ภควนฺตํ ปิฏฺฐิโต ปิฏฺฐิโต อนุพนฺธา โหนฺติ ภิกฺขุสํฆญฺจ ฯ

\hypertarget{uxe2auxe35uxe25}{%
\subsection{สีล}\label{uxe2auxe35uxe25}}

เอวมฺเม สุตํ ฯ เอกํ สมยํ ภควา อนฺตรา จ ราชคหํ อนฺตรา จ นาฬนฺทํ อทฺธานมคฺคปฏิปนฺโน โหติ มหตา ภิกฺขุสํเฆน สทฺธึ ปญฺจมตฺเตหิ ภิกฺขุสเตหิ ฯ สุปฺปิโยปิ โข ปริพฺพาชโก อนฺตรา จ ราชคหํ อนฺตรา จ นาฬนฺทํ อทฺธานมคฺคปฏิปนฺโน โหติ สทฺธึ อนฺเตวาสินา พฺรหฺมทตฺเตน มาณเวน ฯ ตตฺร สุทํ สุปฺปิโย ปริพฺพาชโก อเนกปริยาเยน พุทฺธสฺส อวณฺณํ ภาสติ ธมฺมสฺส อวณฺณํ ภาสติ สํฆสฺส อวณฺณํ ภาสติ ฯ สุปฺปิยสฺส ปน ปริพฺพาชกสฺส อนฺเตวาสี พฺรหฺมทตฺโต มาณโว อเนกปริยาเยน พุทฺธสฺส วณฺณํ ภาสติ ธมฺมสฺส วณฺณํ ภาสติ สํฆสฺส วณฺณํ ภาสติ ฯ อิติห เต อุโภ อาจริยนฺเตวาสี อญฺญมญฺญสฺส อุชุวิปจฺจนิกวาทา ภควนฺตํ ปิฏฺฐิโต ปิฏฺฐิโต อนุพนฺธา โหนฺติ ภิกฺขุสํฆญฺจ ฯ

\hypertarget{uxe08uxe39uxe2cuxe2auxe35uxe25uxe4d}{%
\subsubsection{จูฬสีลํ}\label{uxe08uxe39uxe2cuxe2auxe35uxe25uxe4d}}

เอวมฺเม สุตํ ฯ เอกํ สมยํ ภควา อนฺตรา จ ราชคหํ อนฺตรา จ นาฬนฺทํ อทฺธานมคฺคปฏิปนฺโน โหติ มหตา ภิกฺขุสํเฆน สทฺธึ ปญฺจมตฺเตหิ ภิกฺขุสเตหิ ฯ สุปฺปิโยปิ โข ปริพฺพาชโก อนฺตรา จ ราชคหํ อนฺตรา จ นาฬนฺทํ อทฺธานมคฺคปฏิปนฺโน โหติ สทฺธึ อนฺเตวาสินา พฺรหฺมทตฺเตน มาณเวน ฯ ตตฺร สุทํ สุปฺปิโย ปริพฺพาชโก อเนกปริยาเยน พุทฺธสฺส อวณฺณํ ภาสติ ธมฺมสฺส อวณฺณํ ภาสติ สํฆสฺส อวณฺณํ ภาสติ ฯ สุปฺปิยสฺส ปน ปริพฺพาชกสฺส อนฺเตวาสี พฺรหฺมทตฺโต มาณโว อเนกปริยาเยน พุทฺธสฺส วณฺณํ ภาสติ ธมฺมสฺส วณฺณํ ภาสติ สํฆสฺส วณฺณํ ภาสติ ฯ อิติห เต อุโภ อาจริยนฺเตวาสี อญฺญมญฺญสฺส อุชุวิปจฺจนิกวาทา ภควนฺตํ ปิฏฺฐิโต ปิฏฺฐิโต อนุพนฺธา โหนฺติ ภิกฺขุสํฆญฺจ ฯ

\hypertarget{uxe21uxe0auxe3auxe0cuxe34uxe21uxe2auxe35uxe25uxe4d}{%
\subsubsection{มชฺฌิมสีลํ}\label{uxe21uxe0auxe3auxe0cuxe34uxe21uxe2auxe35uxe25uxe4d}}

เอวมฺเม สุตํ ฯ เอกํ สมยํ ภควา อนฺตรา จ ราชคหํ อนฺตรา จ นาฬนฺทํ อทฺธานมคฺคปฏิปนฺโน โหติ มหตา ภิกฺขุสํเฆน สทฺธึ ปญฺจมตฺเตหิ ภิกฺขุสเตหิ ฯ สุปฺปิโยปิ โข ปริพฺพาชโก อนฺตรา จ ราชคหํ อนฺตรา จ นาฬนฺทํ อทฺธานมคฺคปฏิปนฺโน โหติ สทฺธึ อนฺเตวาสินา พฺรหฺมทตฺเตน มาณเวน ฯ ตตฺร สุทํ สุปฺปิโย ปริพฺพาชโก อเนกปริยาเยน พุทฺธสฺส อวณฺณํ ภาสติ ธมฺมสฺส อวณฺณํ ภาสติ สํฆสฺส อวณฺณํ ภาสติ ฯ สุปฺปิยสฺส ปน ปริพฺพาชกสฺส อนฺเตวาสี พฺรหฺมทตฺโต มาณโว อเนกปริยาเยน พุทฺธสฺส วณฺณํ ภาสติ ธมฺมสฺส วณฺณํ ภาสติ สํฆสฺส วณฺณํ ภาสติ ฯ อิติห เต อุโภ อาจริยนฺเตวาสี อญฺญมญฺญสฺส อุชุวิปจฺจนิกวาทา ภควนฺตํ ปิฏฺฐิโต ปิฏฺฐิโต อนุพนฺธา โหนฺติ ภิกฺขุสํฆญฺจ ฯ

\hypertarget{uxe21uxe2buxe32uxe2auxe35uxe25uxe4d}{%
\subsubsection{มหาสีลํ}\label{uxe21uxe2buxe32uxe2auxe35uxe25uxe4d}}

เอวมฺเม สุตํ ฯ เอกํ สมยํ ภควา อนฺตรา จ ราชคหํ อนฺตรา จ นาฬนฺทํ อทฺธานมคฺคปฏิปนฺโน โหติ มหตา ภิกฺขุสํเฆน สทฺธึ ปญฺจมตฺเตหิ ภิกฺขุสเตหิ ฯ สุปฺปิโยปิ โข ปริพฺพาชโก อนฺตรา จ ราชคหํ อนฺตรา จ นาฬนฺทํ อทฺธานมคฺคปฏิปนฺโน โหติ สทฺธึ อนฺเตวาสินา พฺรหฺมทตฺเตน มาณเวน ฯ ตตฺร สุทํ สุปฺปิโย ปริพฺพาชโก อเนกปริยาเยน พุทฺธสฺส อวณฺณํ ภาสติ ธมฺมสฺส อวณฺณํ ภาสติ สํฆสฺส อวณฺณํ ภาสติ ฯ สุปฺปิยสฺส ปน ปริพฺพาชกสฺส อนฺเตวาสี พฺรหฺมทตฺโต มาณโว อเนกปริยาเยน พุทฺธสฺส วณฺณํ ภาสติ ธมฺมสฺส วณฺณํ ภาสติ สํฆสฺส วณฺณํ ภาสติ ฯ อิติห เต อุโภ อาจริยนฺเตวาสี อญฺญมญฺญสฺส อุชุวิปจฺจนิกวาทา ภควนฺตํ ปิฏฺฐิโต ปิฏฺฐิโต อนุพนฺธา โหนฺติ ภิกฺขุสํฆญฺจ ฯ

\hypertarget{uxe17uxe34uxe10uxe3auxe10uxe34}{%
\subsection{ทิฐฺฐิ}\label{uxe17uxe34uxe10uxe3auxe10uxe34}}

เอวมฺเม สุตํ ฯ เอกํ สมยํ ภควา อนฺตรา จ ราชคหํ อนฺตรา จ นาฬนฺทํ อทฺธานมคฺคปฏิปนฺโน โหติ มหตา ภิกฺขุสํเฆน สทฺธึ ปญฺจมตฺเตหิ ภิกฺขุสเตหิ ฯ สุปฺปิโยปิ โข ปริพฺพาชโก อนฺตรา จ ราชคหํ อนฺตรา จ นาฬนฺทํ อทฺธานมคฺคปฏิปนฺโน โหติ สทฺธึ อนฺเตวาสินา พฺรหฺมทตฺเตน มาณเวน ฯ ตตฺร สุทํ สุปฺปิโย ปริพฺพาชโก อเนกปริยาเยน พุทฺธสฺส อวณฺณํ ภาสติ ธมฺมสฺส อวณฺณํ ภาสติ สํฆสฺส อวณฺณํ ภาสติ ฯ สุปฺปิยสฺส ปน ปริพฺพาชกสฺส อนฺเตวาสี พฺรหฺมทตฺโต มาณโว อเนกปริยาเยน พุทฺธสฺส วณฺณํ ภาสติ ธมฺมสฺส วณฺณํ ภาสติ สํฆสฺส วณฺณํ ภาสติ ฯ อิติห เต อุโภ อาจริยนฺเตวาสี อญฺญมญฺญสฺส อุชุวิปจฺจนิกวาทา ภควนฺตํ ปิฏฺฐิโต ปิฏฺฐิโต อนุพนฺธา โหนฺติ ภิกฺขุสํฆญฺจ ฯ

\hypertarget{uxe1buxe38uxe1euxe3auxe1euxe19uxe3auxe15uxe01uxe1buxe3auxe1buxe34uxe01uxe32}{%
\subsubsection{ปุพฺพนฺตกปฺปิกา}\label{uxe1buxe38uxe1euxe3auxe1euxe19uxe3auxe15uxe01uxe1buxe3auxe1buxe34uxe01uxe32}}

เอวมฺเม สุตํ ฯ เอกํ สมยํ ภควา อนฺตรา จ ราชคหํ อนฺตรา จ นาฬนฺทํ อทฺธานมคฺคปฏิปนฺโน โหติ มหตา ภิกฺขุสํเฆน สทฺธึ ปญฺจมตฺเตหิ ภิกฺขุสเตหิ ฯ สุปฺปิโยปิ โข ปริพฺพาชโก อนฺตรา จ ราชคหํ อนฺตรา จ นาฬนฺทํ อทฺธานมคฺคปฏิปนฺโน โหติ สทฺธึ อนฺเตวาสินา พฺรหฺมทตฺเตน มาณเวน ฯ ตตฺร สุทํ สุปฺปิโย ปริพฺพาชโก อเนกปริยาเยน พุทฺธสฺส อวณฺณํ ภาสติ ธมฺมสฺส อวณฺณํ ภาสติ สํฆสฺส อวณฺณํ ภาสติ ฯ สุปฺปิยสฺส ปน ปริพฺพาชกสฺส อนฺเตวาสี พฺรหฺมทตฺโต มาณโว อเนกปริยาเยน พุทฺธสฺส วณฺณํ ภาสติ ธมฺมสฺส วณฺณํ ภาสติ สํฆสฺส วณฺณํ ภาสติ ฯ อิติห เต อุโภ อาจริยนฺเตวาสี อญฺญมญฺญสฺส อุชุวิปจฺจนิกวาทา ภควนฺตํ ปิฏฺฐิโต ปิฏฺฐิโต อนุพนฺธา โหนฺติ ภิกฺขุสํฆญฺจ ฯ

\hypertarget{uxe2auxe2auxe3auxe2auxe15uxe27uxe32uxe42uxe17}{%
\paragraph{สสฺสตวาโท}\label{uxe2auxe2auxe3auxe2auxe15uxe27uxe32uxe42uxe17}}

เอวมฺเม สุตํ ฯ เอกํ สมยํ ภควา อนฺตรา จ ราชคหํ อนฺตรา จ นาฬนฺทํ อทฺธานมคฺคปฏิปนฺโน โหติ มหตา ภิกฺขุสํเฆน สทฺธึ ปญฺจมตฺเตหิ ภิกฺขุสเตหิ ฯ สุปฺปิโยปิ โข ปริพฺพาชโก อนฺตรา จ ราชคหํ อนฺตรา จ นาฬนฺทํ อทฺธานมคฺคปฏิปนฺโน โหติ สทฺธึ อนฺเตวาสินา พฺรหฺมทตฺเตน มาณเวน ฯ ตตฺร สุทํ สุปฺปิโย ปริพฺพาชโก อเนกปริยาเยน พุทฺธสฺส อวณฺณํ ภาสติ ธมฺมสฺส อวณฺณํ ภาสติ สํฆสฺส อวณฺณํ ภาสติ ฯ สุปฺปิยสฺส ปน ปริพฺพาชกสฺส อนฺเตวาสี พฺรหฺมทตฺโต มาณโว อเนกปริยาเยน พุทฺธสฺส วณฺณํ ภาสติ ธมฺมสฺส วณฺณํ ภาสติ สํฆสฺส วณฺณํ ภาสติ ฯ อิติห เต อุโภ อาจริยนฺเตวาสี อญฺญมญฺญสฺส อุชุวิปจฺจนิกวาทา ภควนฺตํ ปิฏฺฐิโต ปิฏฺฐิโต อนุพนฺธา โหนฺติ ภิกฺขุสํฆญฺจ ฯ

\hypertarget{uxe40uxe2duxe01uxe08uxe3auxe08uxe2auxe2auxe3auxe2auxe15uxe27uxe32uxe42uxe17}{%
\paragraph{เอกจฺจสสฺสตวาโท}\label{uxe40uxe2duxe01uxe08uxe3auxe08uxe2auxe2auxe3auxe2auxe15uxe27uxe32uxe42uxe17}}

เอวมฺเม สุตํ ฯ เอกํ สมยํ ภควา อนฺตรา จ ราชคหํ อนฺตรา จ นาฬนฺทํ อทฺธานมคฺคปฏิปนฺโน โหติ มหตา ภิกฺขุสํเฆน สทฺธึ ปญฺจมตฺเตหิ ภิกฺขุสเตหิ ฯ สุปฺปิโยปิ โข ปริพฺพาชโก อนฺตรา จ ราชคหํ อนฺตรา จ นาฬนฺทํ อทฺธานมคฺคปฏิปนฺโน โหติ สทฺธึ อนฺเตวาสินา พฺรหฺมทตฺเตน มาณเวน ฯ ตตฺร สุทํ สุปฺปิโย ปริพฺพาชโก อเนกปริยาเยน พุทฺธสฺส อวณฺณํ ภาสติ ธมฺมสฺส อวณฺณํ ภาสติ สํฆสฺส อวณฺณํ ภาสติ ฯ สุปฺปิยสฺส ปน ปริพฺพาชกสฺส อนฺเตวาสี พฺรหฺมทตฺโต มาณโว อเนกปริยาเยน พุทฺธสฺส วณฺณํ ภาสติ ธมฺมสฺส วณฺณํ ภาสติ สํฆสฺส วณฺณํ ภาสติ ฯ อิติห เต อุโภ อาจริยนฺเตวาสี อญฺญมญฺญสฺส อุชุวิปจฺจนิกวาทา ภควนฺตํ ปิฏฺฐิโต ปิฏฺฐิโต อนุพนฺธา โหนฺติ ภิกฺขุสํฆญฺจ ฯ

\hypertarget{uxe2duxe19uxe3auxe15uxe32uxe19uxe19uxe3auxe15uxe27uxe32uxe42uxe17}{%
\paragraph{อนฺตานนฺตวาโท}\label{uxe2duxe19uxe3auxe15uxe32uxe19uxe19uxe3auxe15uxe27uxe32uxe42uxe17}}

เอวมฺเม สุตํ ฯ เอกํ สมยํ ภควา อนฺตรา จ ราชคหํ อนฺตรา จ นาฬนฺทํ อทฺธานมคฺคปฏิปนฺโน โหติ มหตา ภิกฺขุสํเฆน สทฺธึ ปญฺจมตฺเตหิ ภิกฺขุสเตหิ ฯ สุปฺปิโยปิ โข ปริพฺพาชโก อนฺตรา จ ราชคหํ อนฺตรา จ นาฬนฺทํ อทฺธานมคฺคปฏิปนฺโน โหติ สทฺธึ อนฺเตวาสินา พฺรหฺมทตฺเตน มาณเวน ฯ ตตฺร สุทํ สุปฺปิโย ปริพฺพาชโก อเนกปริยาเยน พุทฺธสฺส อวณฺณํ ภาสติ ธมฺมสฺส อวณฺณํ ภาสติ สํฆสฺส อวณฺณํ ภาสติ ฯ สุปฺปิยสฺส ปน ปริพฺพาชกสฺส อนฺเตวาสี พฺรหฺมทตฺโต มาณโว อเนกปริยาเยน พุทฺธสฺส วณฺณํ ภาสติ ธมฺมสฺส วณฺณํ ภาสติ สํฆสฺส วณฺณํ ภาสติ ฯ อิติห เต อุโภ อาจริยนฺเตวาสี อญฺญมญฺญสฺส อุชุวิปจฺจนิกวาทา ภควนฺตํ ปิฏฺฐิโต ปิฏฺฐิโต อนุพนฺธา โหนฺติ ภิกฺขุสํฆญฺจ ฯ

\hypertarget{uxe2duxe21uxe23uxe32uxe27uxe34uxe01uxe3auxe40uxe02uxe1buxe27uxe32uxe42uxe17}{%
\paragraph{อมราวิกฺเขปวาโท}\label{uxe2duxe21uxe23uxe32uxe27uxe34uxe01uxe3auxe40uxe02uxe1buxe27uxe32uxe42uxe17}}

เอวมฺเม สุตํ ฯ เอกํ สมยํ ภควา อนฺตรา จ ราชคหํ อนฺตรา จ นาฬนฺทํ อทฺธานมคฺคปฏิปนฺโน โหติ มหตา ภิกฺขุสํเฆน สทฺธึ ปญฺจมตฺเตหิ ภิกฺขุสเตหิ ฯ สุปฺปิโยปิ โข ปริพฺพาชโก อนฺตรา จ ราชคหํ อนฺตรา จ นาฬนฺทํ อทฺธานมคฺคปฏิปนฺโน โหติ สทฺธึ อนฺเตวาสินา พฺรหฺมทตฺเตน มาณเวน ฯ ตตฺร สุทํ สุปฺปิโย ปริพฺพาชโก อเนกปริยาเยน พุทฺธสฺส อวณฺณํ ภาสติ ธมฺมสฺส อวณฺณํ ภาสติ สํฆสฺส อวณฺณํ ภาสติ ฯ สุปฺปิยสฺส ปน ปริพฺพาชกสฺส อนฺเตวาสี พฺรหฺมทตฺโต มาณโว อเนกปริยาเยน พุทฺธสฺส วณฺณํ ภาสติ ธมฺมสฺส วณฺณํ ภาสติ สํฆสฺส วณฺณํ ภาสติ ฯ อิติห เต อุโภ อาจริยนฺเตวาสี อญฺญมญฺญสฺส อุชุวิปจฺจนิกวาทา ภควนฺตํ ปิฏฺฐิโต ปิฏฺฐิโต อนุพนฺธา โหนฺติ ภิกฺขุสํฆญฺจ ฯ

\hypertarget{uxe2duxe18uxe34uxe08uxe3auxe08uxe2auxe21uxe38uxe1buxe3auxe1buxe19uxe3auxe19uxe27uxe32uxe42uxe17}{%
\paragraph{อธิจฺจสมุปฺปนฺนวาโท}\label{uxe2duxe18uxe34uxe08uxe3auxe08uxe2auxe21uxe38uxe1buxe3auxe1buxe19uxe3auxe19uxe27uxe32uxe42uxe17}}

เอวมฺเม สุตํ ฯ เอกํ สมยํ ภควา อนฺตรา จ ราชคหํ อนฺตรา จ นาฬนฺทํ อทฺธานมคฺคปฏิปนฺโน โหติ มหตา ภิกฺขุสํเฆน สทฺธึ ปญฺจมตฺเตหิ ภิกฺขุสเตหิ ฯ สุปฺปิโยปิ โข ปริพฺพาชโก อนฺตรา จ ราชคหํ อนฺตรา จ นาฬนฺทํ อทฺธานมคฺคปฏิปนฺโน โหติ สทฺธึ อนฺเตวาสินา พฺรหฺมทตฺเตน มาณเวน ฯ ตตฺร สุทํ สุปฺปิโย ปริพฺพาชโก อเนกปริยาเยน พุทฺธสฺส อวณฺณํ ภาสติ ธมฺมสฺส อวณฺณํ ภาสติ สํฆสฺส อวณฺณํ ภาสติ ฯ สุปฺปิยสฺส ปน ปริพฺพาชกสฺส อนฺเตวาสี พฺรหฺมทตฺโต มาณโว อเนกปริยาเยน พุทฺธสฺส วณฺณํ ภาสติ ธมฺมสฺส วณฺณํ ภาสติ สํฆสฺส วณฺณํ ภาสติ ฯ อิติห เต อุโภ อาจริยนฺเตวาสี อญฺญมญฺญสฺส อุชุวิปจฺจนิกวาทา ภควนฺตํ ปิฏฺฐิโต ปิฏฺฐิโต อนุพนฺธา โหนฺติ ภิกฺขุสํฆญฺจ ฯ

\hypertarget{uxe2duxe1buxe23uxe19uxe3auxe15uxe01uxe1buxe3auxe1buxe34uxe01uxe32}{%
\subsubsection{อปรนฺตกปฺปิกา}\label{uxe2duxe1buxe23uxe19uxe3auxe15uxe01uxe1buxe3auxe1buxe34uxe01uxe32}}

เอวมฺเม สุตํ ฯ เอกํ สมยํ ภควา อนฺตรา จ ราชคหํ อนฺตรา จ นาฬนฺทํ อทฺธานมคฺคปฏิปนฺโน โหติ มหตา ภิกฺขุสํเฆน สทฺธึ ปญฺจมตฺเตหิ ภิกฺขุสเตหิ ฯ สุปฺปิโยปิ โข ปริพฺพาชโก อนฺตรา จ ราชคหํ อนฺตรา จ นาฬนฺทํ อทฺธานมคฺคปฏิปนฺโน โหติ สทฺธึ อนฺเตวาสินา พฺรหฺมทตฺเตน มาณเวน ฯ ตตฺร สุทํ สุปฺปิโย ปริพฺพาชโก อเนกปริยาเยน พุทฺธสฺส อวณฺณํ ภาสติ ธมฺมสฺส อวณฺณํ ภาสติ สํฆสฺส อวณฺณํ ภาสติ ฯ สุปฺปิยสฺส ปน ปริพฺพาชกสฺส อนฺเตวาสี พฺรหฺมทตฺโต มาณโว อเนกปริยาเยน พุทฺธสฺส วณฺณํ ภาสติ ธมฺมสฺส วณฺณํ ภาสติ สํฆสฺส วณฺณํ ภาสติ ฯ อิติห เต อุโภ อาจริยนฺเตวาสี อญฺญมญฺญสฺส อุชุวิปจฺจนิกวาทา ภควนฺตํ ปิฏฺฐิโต ปิฏฺฐิโต อนุพนฺธา โหนฺติ ภิกฺขุสํฆญฺจ ฯ

\hypertarget{uxe2auxe0duxe3auxe0duxe35uxe27uxe32uxe42uxe17}{%
\paragraph{สญฺญีวาโท}\label{uxe2auxe0duxe3auxe0duxe35uxe27uxe32uxe42uxe17}}

เอวมฺเม สุตํ ฯ เอกํ สมยํ ภควา อนฺตรา จ ราชคหํ อนฺตรา จ นาฬนฺทํ อทฺธานมคฺคปฏิปนฺโน โหติ มหตา ภิกฺขุสํเฆน สทฺธึ ปญฺจมตฺเตหิ ภิกฺขุสเตหิ ฯ สุปฺปิโยปิ โข ปริพฺพาชโก อนฺตรา จ ราชคหํ อนฺตรา จ นาฬนฺทํ อทฺธานมคฺคปฏิปนฺโน โหติ สทฺธึ อนฺเตวาสินา พฺรหฺมทตฺเตน มาณเวน ฯ ตตฺร สุทํ สุปฺปิโย ปริพฺพาชโก อเนกปริยาเยน พุทฺธสฺส อวณฺณํ ภาสติ ธมฺมสฺส อวณฺณํ ภาสติ สํฆสฺส อวณฺณํ ภาสติ ฯ สุปฺปิยสฺส ปน ปริพฺพาชกสฺส อนฺเตวาสี พฺรหฺมทตฺโต มาณโว อเนกปริยาเยน พุทฺธสฺส วณฺณํ ภาสติ ธมฺมสฺส วณฺณํ ภาสติ สํฆสฺส วณฺณํ ภาสติ ฯ อิติห เต อุโภ อาจริยนฺเตวาสี อญฺญมญฺญสฺส อุชุวิปจฺจนิกวาทา ภควนฺตํ ปิฏฺฐิโต ปิฏฺฐิโต อนุพนฺธา โหนฺติ ภิกฺขุสํฆญฺจ ฯ

\hypertarget{uxe2duxe2auxe0duxe3auxe0duxe35uxe27uxe32uxe42uxe17}{%
\paragraph{อสญฺญีวาโท}\label{uxe2duxe2auxe0duxe3auxe0duxe35uxe27uxe32uxe42uxe17}}

เอวมฺเม สุตํ ฯ เอกํ สมยํ ภควา อนฺตรา จ ราชคหํ อนฺตรา จ นาฬนฺทํ อทฺธานมคฺคปฏิปนฺโน โหติ มหตา ภิกฺขุสํเฆน สทฺธึ ปญฺจมตฺเตหิ ภิกฺขุสเตหิ ฯ สุปฺปิโยปิ โข ปริพฺพาชโก อนฺตรา จ ราชคหํ อนฺตรา จ นาฬนฺทํ อทฺธานมคฺคปฏิปนฺโน โหติ สทฺธึ อนฺเตวาสินา พฺรหฺมทตฺเตน มาณเวน ฯ ตตฺร สุทํ สุปฺปิโย ปริพฺพาชโก อเนกปริยาเยน พุทฺธสฺส อวณฺณํ ภาสติ ธมฺมสฺส อวณฺณํ ภาสติ สํฆสฺส อวณฺณํ ภาสติ ฯ สุปฺปิยสฺส ปน ปริพฺพาชกสฺส อนฺเตวาสี พฺรหฺมทตฺโต มาณโว อเนกปริยาเยน พุทฺธสฺส วณฺณํ ภาสติ ธมฺมสฺส วณฺณํ ภาสติ สํฆสฺส วณฺณํ ภาสติ ฯ อิติห เต อุโภ อาจริยนฺเตวาสี อญฺญมญฺญสฺส อุชุวิปจฺจนิกวาทา ภควนฺตํ ปิฏฺฐิโต ปิฏฺฐิโต อนุพนฺธา โหนฺติ ภิกฺขุสํฆญฺจ ฯ

\hypertarget{uxe40uxe19uxe27uxe2auxe0duxe3auxe0duxe35uxe19uxe32uxe2auxe0duxe3auxe0duxe35uxe27uxe32uxe42uxe17}{%
\paragraph{เนวสญฺญีนาสญฺญีวาโท}\label{uxe40uxe19uxe27uxe2auxe0duxe3auxe0duxe35uxe19uxe32uxe2auxe0duxe3auxe0duxe35uxe27uxe32uxe42uxe17}}

เอวมฺเม สุตํ ฯ เอกํ สมยํ ภควา อนฺตรา จ ราชคหํ อนฺตรา จ นาฬนฺทํ อทฺธานมคฺคปฏิปนฺโน โหติ มหตา ภิกฺขุสํเฆน สทฺธึ ปญฺจมตฺเตหิ ภิกฺขุสเตหิ ฯ สุปฺปิโยปิ โข ปริพฺพาชโก อนฺตรา จ ราชคหํ อนฺตรา จ นาฬนฺทํ อทฺธานมคฺคปฏิปนฺโน โหติ สทฺธึ อนฺเตวาสินา พฺรหฺมทตฺเตน มาณเวน ฯ ตตฺร สุทํ สุปฺปิโย ปริพฺพาชโก อเนกปริยาเยน พุทฺธสฺส อวณฺณํ ภาสติ ธมฺมสฺส อวณฺณํ ภาสติ สํฆสฺส อวณฺณํ ภาสติ ฯ สุปฺปิยสฺส ปน ปริพฺพาชกสฺส อนฺเตวาสี พฺรหฺมทตฺโต มาณโว อเนกปริยาเยน พุทฺธสฺส วณฺณํ ภาสติ ธมฺมสฺส วณฺณํ ภาสติ สํฆสฺส วณฺณํ ภาสติ ฯ อิติห เต อุโภ อาจริยนฺเตวาสี อญฺญมญฺญสฺส อุชุวิปจฺจนิกวาทา ภควนฺตํ ปิฏฺฐิโต ปิฏฺฐิโต อนุพนฺธา โหนฺติ ภิกฺขุสํฆญฺจ ฯ

\hypertarget{uxe2duxe38uxe08uxe3auxe40uxe09uxe17uxe27uxe32uxe42uxe17}{%
\paragraph{อุจฺเฉทวาโท}\label{uxe2duxe38uxe08uxe3auxe40uxe09uxe17uxe27uxe32uxe42uxe17}}

เอวมฺเม สุตํ ฯ เอกํ สมยํ ภควา อนฺตรา จ ราชคหํ อนฺตรา จ นาฬนฺทํ อทฺธานมคฺคปฏิปนฺโน โหติ มหตา ภิกฺขุสํเฆน สทฺธึ ปญฺจมตฺเตหิ ภิกฺขุสเตหิ ฯ สุปฺปิโยปิ โข ปริพฺพาชโก อนฺตรา จ ราชคหํ อนฺตรา จ นาฬนฺทํ อทฺธานมคฺคปฏิปนฺโน โหติ สทฺธึ อนฺเตวาสินา พฺรหฺมทตฺเตน มาณเวน ฯ ตตฺร สุทํ สุปฺปิโย ปริพฺพาชโก อเนกปริยาเยน พุทฺธสฺส อวณฺณํ ภาสติ ธมฺมสฺส อวณฺณํ ภาสติ สํฆสฺส อวณฺณํ ภาสติ ฯ สุปฺปิยสฺส ปน ปริพฺพาชกสฺส อนฺเตวาสี พฺรหฺมทตฺโต มาณโว อเนกปริยาเยน พุทฺธสฺส วณฺณํ ภาสติ ธมฺมสฺส วณฺณํ ภาสติ สํฆสฺส วณฺณํ ภาสติ ฯ อิติห เต อุโภ อาจริยนฺเตวาสี อญฺญมญฺญสฺส อุชุวิปจฺจนิกวาทา ภควนฺตํ ปิฏฺฐิโต ปิฏฺฐิโต อนุพนฺธา โหนฺติ ภิกฺขุสํฆญฺจ ฯ

\hypertarget{uxe17uxe34uxe0fuxe3auxe10uxe18uxe21uxe3auxe21uxe19uxe34uxe1euxe3auxe1euxe32uxe19uxe27uxe32uxe42uxe17}{%
\paragraph{ทิฏฺฐธมฺมนิพฺพานวาโท}\label{uxe17uxe34uxe0fuxe3auxe10uxe18uxe21uxe3auxe21uxe19uxe34uxe1euxe3auxe1euxe32uxe19uxe27uxe32uxe42uxe17}}

เอวมฺเม สุตํ ฯ เอกํ สมยํ ภควา อนฺตรา จ ราชคหํ อนฺตรา จ นาฬนฺทํ อทฺธานมคฺคปฏิปนฺโน โหติ มหตา ภิกฺขุสํเฆน สทฺธึ ปญฺจมตฺเตหิ ภิกฺขุสเตหิ ฯ สุปฺปิโยปิ โข ปริพฺพาชโก อนฺตรา จ ราชคหํ อนฺตรา จ นาฬนฺทํ อทฺธานมคฺคปฏิปนฺโน โหติ สทฺธึ อนฺเตวาสินา พฺรหฺมทตฺเตน มาณเวน ฯ ตตฺร สุทํ สุปฺปิโย ปริพฺพาชโก อเนกปริยาเยน พุทฺธสฺส อวณฺณํ ภาสติ ธมฺมสฺส อวณฺณํ ภาสติ สํฆสฺส อวณฺณํ ภาสติ ฯ สุปฺปิยสฺส ปน ปริพฺพาชกสฺส อนฺเตวาสี พฺรหฺมทตฺโต มาณโว อเนกปริยาเยน พุทฺธสฺส วณฺณํ ภาสติ ธมฺมสฺส วณฺณํ ภาสติ สํฆสฺส วณฺณํ ภาสติ ฯ อิติห เต อุโภ อาจริยนฺเตวาสี อญฺญมญฺญสฺส อุชุวิปจฺจนิกวาทา ภควนฺตํ ปิฏฺฐิโต ปิฏฺฐิโต อนุพนฺธา โหนฺติ ภิกฺขุสํฆญฺจ ฯ

\hypertarget{uxe2duxe15uxe3auxe15uxe32uxe42uxe25uxe01uxe1buxe0duxe3auxe0duxe15uxe3auxe15uxe34uxe27uxe15uxe3auxe16uxe38}{%
\subsection{อตฺตาโลกปญฺญตฺติวตฺถุ}\label{uxe2duxe15uxe3auxe15uxe32uxe42uxe25uxe01uxe1buxe0duxe3auxe0duxe15uxe3auxe15uxe34uxe27uxe15uxe3auxe16uxe38}}

เอวมฺเม สุตํ ฯ เอกํ สมยํ ภควา อนฺตรา จ ราชคหํ อนฺตรา จ นาฬนฺทํ อทฺธานมคฺคปฏิปนฺโน โหติ มหตา ภิกฺขุสํเฆน สทฺธึ ปญฺจมตฺเตหิ ภิกฺขุสเตหิ ฯ สุปฺปิโยปิ โข ปริพฺพาชโก อนฺตรา จ ราชคหํ อนฺตรา จ นาฬนฺทํ อทฺธานมคฺคปฏิปนฺโน โหติ สทฺธึ อนฺเตวาสินา พฺรหฺมทตฺเตน มาณเวน ฯ ตตฺร สุทํ สุปฺปิโย ปริพฺพาชโก อเนกปริยาเยน พุทฺธสฺส อวณฺณํ ภาสติ ธมฺมสฺส อวณฺณํ ภาสติ สํฆสฺส อวณฺณํ ภาสติ ฯ สุปฺปิยสฺส ปน ปริพฺพาชกสฺส อนฺเตวาสี พฺรหฺมทตฺโต มาณโว อเนกปริยาเยน พุทฺธสฺส วณฺณํ ภาสติ ธมฺมสฺส วณฺณํ ภาสติ สํฆสฺส วณฺณํ ภาสติ ฯ อิติห เต อุโภ อาจริยนฺเตวาสี อญฺญมญฺญสฺส อุชุวิปจฺจนิกวาทา ภควนฺตํ ปิฏฺฐิโต ปิฏฺฐิโต อนุพนฺธา โหนฺติ ภิกฺขุสํฆญฺจ ฯ

\hypertarget{uxe1buxe23uxe34uxe15uxe2auxe3auxe2auxe34uxe15uxe27uxe34uxe1buxe3auxe1cuxe19uxe3auxe17uxe34uxe15uxe27uxe32uxe42uxe23}{%
\subsubsection{ปริตสฺสิตวิปฺผนฺทิตวาโร}\label{uxe1buxe23uxe34uxe15uxe2auxe3auxe2auxe34uxe15uxe27uxe34uxe1buxe3auxe1cuxe19uxe3auxe17uxe34uxe15uxe27uxe32uxe42uxe23}}

เอวมฺเม สุตํ ฯ เอกํ สมยํ ภควา อนฺตรา จ ราชคหํ อนฺตรา จ นาฬนฺทํ อทฺธานมคฺคปฏิปนฺโน โหติ มหตา ภิกฺขุสํเฆน สทฺธึ ปญฺจมตฺเตหิ ภิกฺขุสเตหิ ฯ สุปฺปิโยปิ โข ปริพฺพาชโก อนฺตรา จ ราชคหํ อนฺตรา จ นาฬนฺทํ อทฺธานมคฺคปฏิปนฺโน โหติ สทฺธึ อนฺเตวาสินา พฺรหฺมทตฺเตน มาณเวน ฯ ตตฺร สุทํ สุปฺปิโย ปริพฺพาชโก อเนกปริยาเยน พุทฺธสฺส อวณฺณํ ภาสติ ธมฺมสฺส อวณฺณํ ภาสติ สํฆสฺส อวณฺณํ ภาสติ ฯ สุปฺปิยสฺส ปน ปริพฺพาชกสฺส อนฺเตวาสี พฺรหฺมทตฺโต มาณโว อเนกปริยาเยน พุทฺธสฺส วณฺณํ ภาสติ ธมฺมสฺส วณฺณํ ภาสติ สํฆสฺส วณฺณํ ภาสติ ฯ อิติห เต อุโภ อาจริยนฺเตวาสี อญฺญมญฺญสฺส อุชุวิปจฺจนิกวาทา ภควนฺตํ ปิฏฺฐิโต ปิฏฺฐิโต อนุพนฺธา โหนฺติ ภิกฺขุสํฆญฺจ ฯ

\hypertarget{uxe1cuxe2auxe3auxe2auxe1buxe08uxe3auxe08uxe22uxe32uxe27uxe32uxe42uxe23}{%
\subsubsection{ผสฺสปจฺจยาวาโร}\label{uxe1cuxe2auxe3auxe2auxe1buxe08uxe3auxe08uxe22uxe32uxe27uxe32uxe42uxe23}}

เอวมฺเม สุตํ ฯ เอกํ สมยํ ภควา อนฺตรา จ ราชคหํ อนฺตรา จ นาฬนฺทํ อทฺธานมคฺคปฏิปนฺโน โหติ มหตา ภิกฺขุสํเฆน สทฺธึ ปญฺจมตฺเตหิ ภิกฺขุสเตหิ ฯ สุปฺปิโยปิ โข ปริพฺพาชโก อนฺตรา จ ราชคหํ อนฺตรา จ นาฬนฺทํ อทฺธานมคฺคปฏิปนฺโน โหติ สทฺธึ อนฺเตวาสินา พฺรหฺมทตฺเตน มาณเวน ฯ ตตฺร สุทํ สุปฺปิโย ปริพฺพาชโก อเนกปริยาเยน พุทฺธสฺส อวณฺณํ ภาสติ ธมฺมสฺส อวณฺณํ ภาสติ สํฆสฺส อวณฺณํ ภาสติ ฯ สุปฺปิยสฺส ปน ปริพฺพาชกสฺส อนฺเตวาสี พฺรหฺมทตฺโต มาณโว อเนกปริยาเยน พุทฺธสฺส วณฺณํ ภาสติ ธมฺมสฺส วณฺณํ ภาสติ สํฆสฺส วณฺณํ ภาสติ ฯ อิติห เต อุโภ อาจริยนฺเตวาสี อญฺญมญฺญสฺส อุชุวิปจฺจนิกวาทา ภควนฺตํ ปิฏฺฐิโต ปิฏฺฐิโต อนุพนฺธา โหนฺติ ภิกฺขุสํฆญฺจ ฯ

\hypertarget{uxe40uxe19uxe15uxe4d-uxe10uxe32uxe19uxe4d-uxe27uxe34uxe0auxe3auxe0auxe15uxe34uxe27uxe32uxe42uxe23}{%
\subsubsection{เนตํ ฐานํ วิชฺชติวาโร}\label{uxe40uxe19uxe15uxe4d-uxe10uxe32uxe19uxe4d-uxe27uxe34uxe0auxe3auxe0auxe15uxe34uxe27uxe32uxe42uxe23}}

เอวมฺเม สุตํ ฯ เอกํ สมยํ ภควา อนฺตรา จ ราชคหํ อนฺตรา จ นาฬนฺทํ อทฺธานมคฺคปฏิปนฺโน โหติ มหตา ภิกฺขุสํเฆน สทฺธึ ปญฺจมตฺเตหิ ภิกฺขุสเตหิ ฯ สุปฺปิโยปิ โข ปริพฺพาชโก อนฺตรา จ ราชคหํ อนฺตรา จ นาฬนฺทํ อทฺธานมคฺคปฏิปนฺโน โหติ สทฺธึ อนฺเตวาสินา พฺรหฺมทตฺเตน มาณเวน ฯ ตตฺร สุทํ สุปฺปิโย ปริพฺพาชโก อเนกปริยาเยน พุทฺธสฺส อวณฺณํ ภาสติ ธมฺมสฺส อวณฺณํ ภาสติ สํฆสฺส อวณฺณํ ภาสติ ฯ สุปฺปิยสฺส ปน ปริพฺพาชกสฺส อนฺเตวาสี พฺรหฺมทตฺโต มาณโว อเนกปริยาเยน พุทฺธสฺส วณฺณํ ภาสติ ธมฺมสฺส วณฺณํ ภาสติ สํฆสฺส วณฺณํ ภาสติ ฯ อิติห เต อุโภ อาจริยนฺเตวาสี อญฺญมญฺญสฺส อุชุวิปจฺจนิกวาทา ภควนฺตํ ปิฏฺฐิโต ปิฏฺฐิโต อนุพนฺธา โหนฺติ ภิกฺขุสํฆญฺจ ฯ

\hypertarget{uxe17uxe34uxe0fuxe3auxe34uxe10uxe04uxe15uxe34uxe01uxe32uxe18uxe34uxe0fuxe3auxe10uxe32uxe19uxe27uxe0fuxe3auxe0fuxe01uxe16uxe32}{%
\subsubsection{ทิฏฺิฐคติกาธิฏฺฐานวฏฺฏกถา}\label{uxe17uxe34uxe0fuxe3auxe34uxe10uxe04uxe15uxe34uxe01uxe32uxe18uxe34uxe0fuxe3auxe10uxe32uxe19uxe27uxe0fuxe3auxe0fuxe01uxe16uxe32}}

เอวมฺเม สุตํ ฯ เอกํ สมยํ ภควา อนฺตรา จ ราชคหํ อนฺตรา จ นาฬนฺทํ อทฺธานมคฺคปฏิปนฺโน โหติ มหตา ภิกฺขุสํเฆน สทฺธึ ปญฺจมตฺเตหิ ภิกฺขุสเตหิ ฯ สุปฺปิโยปิ โข ปริพฺพาชโก อนฺตรา จ ราชคหํ อนฺตรา จ นาฬนฺทํ อทฺธานมคฺคปฏิปนฺโน โหติ สทฺธึ อนฺเตวาสินา พฺรหฺมทตฺเตน มาณเวน ฯ ตตฺร สุทํ สุปฺปิโย ปริพฺพาชโก อเนกปริยาเยน พุทฺธสฺส อวณฺณํ ภาสติ ธมฺมสฺส อวณฺณํ ภาสติ สํฆสฺส อวณฺณํ ภาสติ ฯ สุปฺปิยสฺส ปน ปริพฺพาชกสฺส อนฺเตวาสี พฺรหฺมทตฺโต มาณโว อเนกปริยาเยน พุทฺธสฺส วณฺณํ ภาสติ ธมฺมสฺส วณฺณํ ภาสติ สํฆสฺส วณฺณํ ภาสติ ฯ อิติห เต อุโภ อาจริยนฺเตวาสี อญฺญมญฺญสฺส อุชุวิปจฺจนิกวาทา ภควนฺตํ ปิฏฺฐิโต ปิฏฺฐิโต อนุพนฺธา โหนฺติ ภิกฺขุสํฆญฺจ ฯ

\hypertarget{dm}{%
\chapter{มหาวคฺคปาฬิ}\label{dm}}

\hypertarget{uxe21uxe2buxe32uxe1buxe17uxe32uxe19uxe2auxe38uxe15uxe3auxe15uxe4d}{%
\section{มหาปทานสุตฺตํ}\label{uxe21uxe2buxe32uxe1buxe17uxe32uxe19uxe2auxe38uxe15uxe3auxe15uxe4d}}

\hypertarget{uxe42uxe1euxe18uxe34uxe2auxe15uxe3auxe15uxe18uxe21uxe3auxe21uxe15uxe32}{%
\subsection{โพธิสตฺตธมฺมตา}\label{uxe42uxe1euxe18uxe34uxe2auxe15uxe3auxe15uxe18uxe21uxe3auxe21uxe15uxe32}}

\hypertarget{uxe17uxe3auxe27uxe15uxe3auxe15uxe36uxe2auxe21uxe2buxe32uxe1buxe38uxe23uxe34uxe2auxe25uxe01uxe3auxe02uxe13uxe32}{%
\subsection{ทฺวตฺตึสมหาปุริสลกฺขณา}\label{uxe17uxe3auxe27uxe15uxe3auxe15uxe36uxe2auxe21uxe2buxe32uxe1buxe38uxe23uxe34uxe2auxe25uxe01uxe3auxe02uxe13uxe32}}

\hypertarget{uxe27uxe34uxe1buxe2auxe3auxe2auxe35uxe2auxe21uxe0duxe3auxe0duxe32}{%
\subsection{วิปสฺสีสมญฺญา}\label{uxe27uxe34uxe1buxe2auxe3auxe2auxe35uxe2auxe21uxe0duxe3auxe0duxe32}}

\hypertarget{uxe0auxe34uxe13uxe3auxe13uxe1buxe38uxe23uxe34uxe42uxe2a}{%
\subsection{ชิณฺณปุริโส}\label{uxe0auxe34uxe13uxe3auxe13uxe1buxe38uxe23uxe34uxe42uxe2a}}

\hypertarget{uxe1euxe3auxe22uxe32uxe18uxe34uxe15uxe1buxe38uxe23uxe34uxe42uxe2a}{%
\subsection{พฺยาธิตปุริโส}\label{uxe1euxe3auxe22uxe32uxe18uxe34uxe15uxe1buxe38uxe23uxe34uxe42uxe2a}}

\hypertarget{uxe01uxe32uxe25uxe07uxe3auxe01uxe15uxe1buxe38uxe23uxe34uxe42uxe2a}{%
\subsection{กาลงฺกตปุริโส}\label{uxe01uxe32uxe25uxe07uxe3auxe01uxe15uxe1buxe38uxe23uxe34uxe42uxe2a}}

\hypertarget{uxe1buxe1euxe3auxe1euxe0auxe34uxe42uxe15}{%
\subsection{ปพฺพชิโต}\label{uxe1buxe1euxe3auxe1euxe0auxe34uxe42uxe15}}

\hypertarget{uxe42uxe1euxe18uxe34uxe2auxe15uxe3auxe15uxe1buxe1euxe3auxe1euxe0auxe3auxe0auxe32}{%
\subsection{โพธิสตฺตปพฺพชฺชา}\label{uxe42uxe1euxe18uxe34uxe2auxe15uxe3auxe15uxe1buxe1euxe3auxe1euxe0auxe3auxe0auxe32}}

\hypertarget{uxe21uxe2buxe32uxe0auxe19uxe01uxe32uxe22uxe2duxe19uxe38uxe1buxe1euxe3auxe1euxe0auxe3auxe0auxe32}{%
\subsection{มหาชนกายอนุปพฺพชฺชา}\label{uxe21uxe2buxe32uxe0auxe19uxe01uxe32uxe22uxe2duxe19uxe38uxe1buxe1euxe3auxe1euxe0auxe3auxe0auxe32}}

\hypertarget{uxe42uxe1euxe18uxe34uxe2auxe15uxe3auxe15uxe2duxe20uxe34uxe19uxe34uxe40uxe27uxe42uxe2a}{%
\subsection{โพธิสตฺตอภินิเวโส}\label{uxe42uxe1euxe18uxe34uxe2auxe15uxe3auxe15uxe2duxe20uxe34uxe19uxe34uxe40uxe27uxe42uxe2a}}

\hypertarget{uxe1euxe3auxe23uxe2buxe3auxe21uxe22uxe32uxe08uxe19uxe01uxe16uxe32}{%
\subsection{พฺรหฺมยาจนกถา}\label{uxe1euxe3auxe23uxe2buxe3auxe21uxe22uxe32uxe08uxe19uxe01uxe16uxe32}}

\hypertarget{uxe2duxe04uxe3auxe04uxe2auxe32uxe27uxe01uxe22uxe38uxe04uxe4d}{%
\subsection{อคฺคสาวกยุคํ}\label{uxe2duxe04uxe3auxe04uxe2auxe32uxe27uxe01uxe22uxe38uxe04uxe4d}}

\hypertarget{uxe21uxe2buxe32uxe0auxe19uxe01uxe32uxe22uxe1buxe1euxe3auxe1euxe0auxe3auxe0auxe32}{%
\subsection{มหาชนกายปพฺพชฺชา}\label{uxe21uxe2buxe32uxe0auxe19uxe01uxe32uxe22uxe1buxe1euxe3auxe1euxe0auxe3auxe0auxe32}}

\hypertarget{uxe1buxe38uxe23uxe34uxe21uxe1buxe1euxe3auxe1euxe0auxe34uxe15uxe32uxe19uxe4d-uxe18uxe21uxe3auxe21uxe32uxe20uxe34uxe2auxe21uxe42uxe22}{%
\subsection{ปุริมปพฺพชิตานํ ธมฺมาภิสมโย}\label{uxe1buxe38uxe23uxe34uxe21uxe1buxe1euxe3auxe1euxe0auxe34uxe15uxe32uxe19uxe4d-uxe18uxe21uxe3auxe21uxe32uxe20uxe34uxe2auxe21uxe42uxe22}}

\hypertarget{uxe08uxe32uxe23uxe34uxe01uxe32uxe2duxe19uxe38uxe0auxe32uxe19uxe19uxe4d}{%
\subsection{จาริกาอนุชานนํ}\label{uxe08uxe32uxe23uxe34uxe01uxe32uxe2duxe19uxe38uxe0auxe32uxe19uxe19uxe4d}}

\hypertarget{uxe40uxe17uxe27uxe15uxe32uxe42uxe23uxe08uxe19uxe4d}{%
\subsection{เทวตาโรจนํ}\label{uxe40uxe17uxe27uxe15uxe32uxe42uxe23uxe08uxe19uxe4d}}

\hypertarget{uxe21uxe2buxe32uxe19uxe34uxe17uxe32uxe19uxe2auxe38uxe15uxe3auxe15uxe4d}{%
\section{มหานิทานสุตฺตํ}\label{uxe21uxe2buxe32uxe19uxe34uxe17uxe32uxe19uxe2auxe38uxe15uxe3auxe15uxe4d}}

\hypertarget{uxe1buxe0fuxe34uxe08uxe3auxe08uxe2auxe21uxe38uxe1buxe3auxe1buxe32uxe42uxe17}{%
\subsection{ปฏิจฺจสมุปฺปาโท}\label{uxe1buxe0fuxe34uxe08uxe3auxe08uxe2auxe21uxe38uxe1buxe3auxe1buxe32uxe42uxe17}}

\hypertarget{uxe2duxe15uxe3auxe15uxe1buxe0duxe3auxe0duxe15uxe3auxe15uxe34}{%
\subsection{อตฺตปญฺญตฺติ}\label{uxe2duxe15uxe3auxe15uxe1buxe0duxe3auxe0duxe15uxe3auxe15uxe34}}

\hypertarget{uxe19uxe2duxe15uxe3auxe15uxe1buxe0duxe3auxe0duxe15uxe3auxe15uxe34}{%
\subsection{นอตฺตปญฺญตฺติ}\label{uxe19uxe2duxe15uxe3auxe15uxe1buxe0duxe3auxe0duxe15uxe3auxe15uxe34}}

\hypertarget{uxe2duxe15uxe3auxe15uxe2auxe21uxe19uxe38uxe1buxe2auxe3auxe2auxe19uxe32}{%
\subsection{อตฺตสมนุปสฺสนา}\label{uxe2duxe15uxe3auxe15uxe2auxe21uxe19uxe38uxe1buxe2auxe3auxe2auxe19uxe32}}

\hypertarget{uxe2auxe15uxe3auxe15-uxe27uxe34uxe3auxe0duxe0duxe32uxe13uxe0fuxe3auxe34uxe0duxe15uxe34}{%
\subsection{สตฺต วิฺญญาณฏฺิญติ}\label{uxe2auxe15uxe3auxe15-uxe27uxe34uxe3auxe0duxe0duxe32uxe13uxe0fuxe3auxe34uxe0duxe15uxe34}}

\hypertarget{uxe2duxe0fuxe3auxe10-uxe27uxe34uxe42uxe21uxe01uxe3auxe02uxe32}{%
\subsection{อฏฺฐ วิโมกฺขา}\label{uxe2duxe0fuxe3auxe10-uxe27uxe34uxe42uxe21uxe01uxe3auxe02uxe32}}

\end{document}
